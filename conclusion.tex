\chapter{Conclusion}
\label{ch:Conclusion}

D2D communication shows itself as a good solution for improving mobile networks, specially when applied for data offloading. However, in D2D networks devices need to share its own resources to forward data to other nodes, which can lead to bad user experience if the proposed solutions does not provide proper mechanisms to prevent overuse of memory and power resources. In this work we proposed two solutions to contribute in solving this problem.

The first one is a new buffer management algorithm for D2D opportunistic networks named Space-Time-Drop (ST-Drop). This algorithm uses the idea that a message with more space and time coverage can be dropped first, because it is more likely that such a message has already been delivered. We have combined the idea of basic drop policies to measure the space and time coverage of a message using local information. Therefore, our solution can be easily deployed in distributed environments. We evaluated our solution with simulations in Epidemic, Prophet, and Bubble Rap routing algorithms using three different traffic loads with two different mobility traces. We showed that our solution achieves higher delivery ratio with the three routing algorithms and much lower overhead values with Prophet and Bubble Rap. Considering all combinations of the evaluated opportunistic routing algorithms and buffer management policies, the ST-Drop combined with Bubble Rap achieved, in all experiments, the best cost-effectiveness, i.e., the highest delivery ratio with the lowest network overhead. This result shows that ST-Drop significantly contributes to improving the cost-effectiveness of opportunistic D2D Networks.

The second contribution is a new version of Groups-NET forwarding algorithm targeting distributed environments. Groups-NET is a forwarding algorithm that explores group encounters regularity to forward messages. Groups-NET does not suffer of the problem of concentrating the network traffic at a group of nodes with higher utility values, because it does not use utility functions at the node level. However the original solution definition uses a centralized algorithm with global network knowledge to detect groups, which makes it unfeasible for distributed environments. So, we proposed an algorithm for group detection and tracking in a distributed environment and combined it with the original Groups-NET forwarding decision engine. Our group detection algorithm captured the group reencounters regularity described in the centralized approach, validating our solution. We executed forwarding simulations using groups detected by the distributed solution as input with a real mobility trace containing 115 nodes, and the solution achieves a good delivery ratio with an expressive lower network overhead when compared with the state-of-art BubbleRap algorithm.

Both ST-Drop and distributed Groups-NET showed good results and provided valuable insights for future works. As a next step both solutions need to be tested in more environments with more nodes and different patterns of data generation. There is also space for combining the two solutions. This combination was not included in this work because it requires an extensive modification of ONE (Opportunistic Network) simulation code, but it can be done in extension works. Regarding ST-Drop we also highlight the possibility of exploring more metrics to measure space and time coverage. Information obtained from the device's context, such as battery information and actual spatial information (GPS), could be used to improve ST-Drop. For distributed Groups-NET there is also the possibility of changing the forwarding decision engine to consider different patterns of group-to-group routes. One possibility is to use the $N$ shortest group paths to forward the message and check if the delivery ratio is improved without impacting the network overhead.