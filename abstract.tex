In D2D multi-hop forwarding protocols nodes need to cooperate and act as relays to messages of other nodes. As a result of this cooperative behavior,
each node needs to share its resources with the network allocating memory to the message in a buffer and also expending some of its power resource to receive and forward it. Therefore, forwarding protocols should achieve a high delivery ratio but using a low number of message transmissions to not impact user experience. In this work we present two contributions to solve this problem.

The first is a new buffer management algorithm for opportunistic routing in D2D networks named ST-Drop (Space-Time-Drop). It basically explores the idea that a message with a higher time and space coverage can be discarded first in situations of buffer overflow. We have evaluated it in three different types of opportunistic routing algorithms: epidemic-based, probabilistic, and social-aware. We have conducted simulations using two different publicly available data sources and considered different network traffic loads. Compared to other message drop policies, ST-Drop obtained the highest message delivery ratio in all considered scenarios and the lowest overhead when applied to the state-of-art social-aware and probabilistic routing algorithms, namely, Bubble Rap and Prophet.

The second contribution is a new version of Groups-NET forwarding algorithm targeting distributed environments. Groups-NET is a forwarding algorithm that explores group encounters regularity to forward messages. The original solution definition uses a centralized algorithm with global network knowledge to detect groups, which makes it unfeasible for distributed environments. We proposed an algorithm for group detection and tracking in a distributed environment and combined it with the original Groups-NET forwarding decision engine. Through experiments with a real mobility trace containing 115 nodes, we show that the solution achieves a good delivery ratio with an expressive lower network overhead when compared with the state-of-art BubbleRap algorithm.