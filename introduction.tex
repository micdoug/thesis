\chapter{Introduction}
\label{ch:Introduction}

\section{Motivation}
\label{sec:intro.motivation}

Mobile devices have become smaller and more powerful over the years. Smartphone based computing and communication has become ubiquitous.
Connected services and applications like social networks, instant messaging, content distribution systems and games, for example, have imposed several traffic growth in the mobile network. One approach to solve this problem is to improve the network infrastructure. In indoor environments this strategy makes sense, because usually they are smaller in number of devices. However in outdoor environments this approach can be unfeasible, because they have a high number of devices that can vary due to people mobility.

Since improving the infrastructure can be insufficient or even unfeasible, researchers have been discussing alternative solutions. The current mobile communication model is driven by base stations, in the sense that devices must contact base stations to communicate with other devices or services. This centralized architecture have imposed a bottleneck on network evolution, so recent works have explored a new communication model in which mobile devices can bypass the base station to communicate directly with near devices. This model is called Device to Device communication or D2D \citep{yang2013solving}, and can be used for offload traffic from base stations \citep*{yang2013solving, nunes2016leveraging, aijaz2013survey, pyattaev2013proximity, andreev2014cellular, bastug2014living}, to provide proximity based services \citep{lin2014overview} and to provide extended network coverage under emergency scenarios \citep{babun2015multi}, for example.

\section{The Problem}
\label{sec:intro.problemStatement}

The D2D communication shows itself as a sensible solution to evolve the mobile network system and improve the devices experience. However the direct communication nature of this model makes the regular Internet communication protocols unapplicable. One of the major reasons for this is the absence of end-to-end path between two nodes, since messages are transmitted when there are contacts between two or more nodes and these contacts are driven by people mobility. This fact by itself makes the IP protocol routing mechanism unapplicable, which makes the routing problem one of the major topics of study \citep{misra2016opportunistic}. Another reason is the intermittently connections caused by nodes mobility, which makes impossible for a node to forward a packet immediately after receiving it, as is expected in the regular IP protocol.

These problems were discussed and solutions were proposed in the context of Disruption Tolerant Networks (DTNs) \citep{fall2003delay}. Regular networks are based on the \textit{store-and-forward} paradigm, in which nodes after receiving a message immediately forward it to other connected node that can help to deliver the message to the destiny. DTN networks define the \textit{store-carry-and-forward} paradigm, in which a node after receiving a message, it stores it in a persistent buffer, and carries it until there is a proper contact to forward the message. Using this model we can forward messages in D2D networks, however we need to provide appropriate routing protocols. As said earlier in this scenario there is no predefined end-to-end path between two nodes, so the regular IP routing mechanism does not work \citep{misra2016opportunistic}. There are several proposals to solve this problem, in which the major solutions are based on flooding variations or utility functions derived from probability theorems or social context exploration.

Routing protocols based on utility functions have shown the best balance between message delivery and number of transmissions. However, the major solutions measure utility functions at the individual level, which leads to traffic concentration on nodes with higher utility values, penalizing them \citep{chilipirea2013energy}. This is a remarkable problem, because in networks that use the \textit{store-carry-and-forward} paradigm nodes need to allocate a buffer to store messages until there is chance to forward it. If the traffic is too high, some nodes can have problems of buffer overflow \citep{silva2015survey}.

\section{Contributions}
\label{sec:intro.objectives}

This work has two major contributions. First we propose a new buffer management strategy called \textit{Space Time Drop} or just ST-Drop that aims to solve the problem of buffer management under high traffic demands. The proposed solution shows a notable performance when combined with social aware and probability based routing algorithms, outperforming classic approaches. The second contribution is the definition of a distributed implementation of a new social aware routing algorithm called Groups-NET proposed in \citep{nunes2016leveraging}. This algorithm does not rely on individual utility functions, which alleviates the problem of traffic concentration on some nodes. We propose a distributed algorithm for detecting and manage groups. The initial experiments show that the algorithm outperforms the BubbleRap algorithm on network overhead metric with compatible delivery ratio.

\section{Structure}
\label{sec:intro.structure}
The work is organized as follows. In chapter \ref{ch:MessageForwarding} we discuss the main approaches and algorithms to forward messages in D2D networks. This chapter introduce important concepts to understand the proposed solutions. In chapter \ref{ch:StDrop} we discuss the classic approaches of buffer management and introduce the proposed algorithm ST-Drop. In chapter \ref{ch:GroupsNet} we introduce the distributed implementation of the Groups-NET algorithm and present some initial results. And in chapter \ref{ch:Conclusion} we conclude the work with a summary of the main results and discuss future works to expand the research.
